\section{Application to Domain Adaptation}

\begin{figure}[h]
  \begin{center}
  \centerline{\includegraphics[width=\columnwidth]{chapters/OTARI/figures/visu_DA.pdf}}
  \caption{Toy domain adaptation scenario with entropic OT plans ($\xi=10$) with various constraints. The size of the point is proportional to the associated entropy. When using Sinkhorn, barycentric mapping match outliers points since the OT plan is less diffuse for these points. In turn, using pointwise constraints concentrate the 
   mapped points in high-density regions, thus giving more robust estimates for the mappings onto the target domain.}
  \label{fig:visu_DA}
  \end{center}
\end{figure}

\begin{table}[h]
  % \begin{wraptable}[7]{L}{8.5cm}
    % \vspace*{-0.4cm}
    \centering
    % Note that unregularized OT does not depend on $\xi$. }
    \begin{small}
    \begin{tabular}{lc@{\hskip 0.1in}c@{\hskip 0.1in}c@{\hskip 0.1in}c@{\hskip 0.1in}c@{\hskip 0.1in}c}
    \toprule[1.5pt]
    & OT& EOT & EOTARI-s & EOTARI-t & EOTARI-d \\
    \midrule
    MNIST $\to$ USPS ($\xi=30$) & $53.1(5.4)$ & $64.2(2.8)$ & $65.0(5.3)$ & $66.4(3.5)$ & $\mathbf{67.4(2.9)}$ \\
    MNIST $\to$ USPS ($\xi=300$) & $53.1(5.4)$ & $68.8(3.1)$ & $70.8(4.2)$ & $70.2(3.4)$ & $\mathbf{72.6(5.1)}$ \\
  
    USPS $\to$ MNIST ($\xi=30$) & $59.1(4.9)$ & $60.8(5.4)$ & $61.6(4.4)$ & $\mathbf{62.6(3.0)}$ & $61.0(4.7)$ \\
    USPS $\to$ MNIST ($\xi=300$) & $59.1(4.9)$ & $59.8(1.6)$ & $61.0(2.3)$ & $\mathbf{61.6(3.0)}$ & $58.8(2.3)$ \\
    \midrule
    \midrule
    & OT & QOT & QOTARI-s & QOTARI-t & QOTARI-d \\
    \midrule
    MNIST $\to$ USPS ($\xi=30$) & $53.1(5.4)$ & $68.3(3.9)$ & $68.3(3.6)$ & $\mathbf{69.3(4.7)}$ & $68.1(4.6)$ \\
    MNIST $\to$ USPS ($\xi=300$) & $53.1(5.4)$ & $60.7(1.5)$ & $\mathbf{67.0(2.4)}$ & $65.5(2.3)$ & $65.8(2.5)$ \\
    USPS $\to$ MNIST ($\xi=30$) & $59.1(4.9)$ & $60.4(3.5)$ & $\mathbf{62.8(3.7)}$ & $59.6(2.7)$ & $61.6(3.1)$ \\
    USPS $\to$ MNIST ($\xi=300$) & $59.1(4.9)$ & $59.2(3.4)$ & $60.1(3.0)$ & $\mathbf{62.0(3.7)}$ & $61.5(3.8)$ \\
    \bottomrule[1.5pt]
    \end{tabular}
    \end{small}
    \vspace{0.4cm}
    \caption{Domain adaptation 1-NN classification scores for OT (unregularized), EOT (entropic), EOTARI (entropic OTARI), QOT (quadratic), QOTARI (quadratic OTARI) for $\xi=30$ and $\xi=300$.}
    \label{tab:da_exps}
    \vspace{0.1cm}
  \end{table}

In this section, we evaluate OTARI on a domain adaptation task where the goal is to transport labeled data points to a target domain where a classifier is trained. Mapping onto the target domain is performed through a barycentric mapping of the form: for any $i \in \integ{N_S}$, $\hat{\bm{x}}_i = \frac{1}{a_i} \sum_j T_{ij} \bm{x}_j^T$.
Looking at \cref{fig:visu_DA}, one can notice that using OTARI for domain adaptation yields a mapping that is concentrated in high-density (thus more faithful) regions of the target domain. On the opposite, when using globally constrained OT (left side of \cref{fig:visu_DA}), the barycentric mapping associated with an outlier is concentrated on the outlier's position. For the experiments, we take $\C$ as the squared Euclidean distance computed from raw images of the handwritten digit classification benchmark MNIST-USPS. 
Following the standard practice in OT-based domain adaptation \cite{flamary2016optimal}, we map the source to the target samples and then train a 1-NN classifier on the barycentric mappings with source labels.
We compute the outcomes across 10 independent trials. In each of these experiments, the target data is partitioned into a 90\% training and 10\% testing split, with OT barycentric mappings and 1-NN classifiers exclusively applied to the training set. Mean scores and standard deviations are displayed in \cref{tab:da_exps}. The latter shows that adaptive regularization consistently outperforms global regularization (set such that $\sum_i \psi(\mP^\star_{i:}) = n \psi(\mathbf{e}_{\xi})$ where $\mP^\star$ solves \ref{eq:cot} for a fair comparison) with significant performance gains in some settings.