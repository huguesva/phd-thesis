% !TeX root = ../workshop_paper.tex


\section{Optimal Transport with Adaptive Regularization}\label{sec:pwot}

In this section, we present a new formulation of OT that imposes constraints on each row of the OT plan.
We begin by introducing the set of matrices with \emph{point-wise} constraints. To set the upper bound, we rely on the perplexity parameter $\xi$ \cite{van2023snekhorn} that can be interpreted as the number of effective neighbors for each point. Concretely, we define $\mathbf{e}_{\xi} = \frac{1}{\xi}(\ind_{i \leq \xi})_{i}$ the probability vector of $\xi$ uniform arms and
\begin{align}
  \mathcal{B}_\psi(\xi) &\coloneqq \{\mP \bm{\geq} \bm{0} \: \text{s.t.}\ \: \forall i, \: \psi(\mP_{i:}) \leq \psi(\mathbf{e}_{\xi}) \} \:.
\end{align}
Note that in particular $\psi_{\KL}(\mathbf{e}_{\xi}) = - (\log \xi + 1)$ and $\psi_2(\mathbf{e}_{\xi}) = 1/\xi$.
% In contrast to the constraint utilized in standard strictly convex OT \eqref{eq:scOT}, $\mathcal{B}_\psi(\xi)$ imposes a constraint on \emph{each row} of the transport plan $\mP$.
We now define Optimal Transport with Adaptive Regularization (OTARI) as the generalization of \eqref{eq:cot} to the case where the strictly convex constraint is given by $\mathcal{B}_\psi(\xi)$. Similarly to \cref{prop:cot} (\cref{sec:proof_global}), we can frame OTARI as a $\psi$-Bregman projection of $\mK_\sigma = \nabla \psi^*(-\mC / \sigma)$ or solve it using dual ascent.

\paragraph{OTARI problem.}
  Let $(\bm{a}, \bm{b}, \xi)$ be such that $\Pi(\bm{a}, \bm{b}) \cap \mathcal{B}_\psi(\xi)$ has an interior point.
  We consider the following problem:
  \begin{align}\label{eq:pcOT}
    \min_{\mP \in \Pi(\bm{a}, \bm{b})} \: \langle \mP, \mC \rangle \quad \text{s.t.} \quad  \mP \in \mathcal{B}_\psi(\xi) \:.
    \tag{OTARI-s}
\end{align}

\paragraph{Mirror descent.}
We propose to solve the above using a mirror descent algorithm that consists in the following update of the primal variable $\mP$:
\begin{align}
  \mP_{t+1} &= \argmin_{\mP \in \Pi(\bm{a}, \bm{b}) \cap \mathcal{B}_\psi(\xi)} \: \scalar{\mP}{\mC} + \frac{1}{\gamma_t} D_{\psi}(\mP, \mP_t) \\
  &= \argmin_{\mP \in \Pi(\bm{a}, \bm{b}) \cap \mathcal{B}_\psi(\xi)} \: \psi(\mP) - \scalar{\mP}{\nabla \psi(\mP_t) - \gamma_t \mC} \\
  &= \operatorname{Proj}^{D_\psi}_{\Pi(\bm{a}, \bm{b}) \cap \mathcal{B}_\psi(\xi)}(\nabla \psi^*(\nabla \psi(\mP_t) - \gamma_t \mC)) \:.
\end{align}

\begin{table}[h]
  \centering
  \caption{Table caption}
  \label{tab:example}
  \begin{tabular}{lccc}
    \toprule
    & $\psi(\mathbf{p})$ & $\nabla \psi(\mathbf{p})$ & $\nabla \psi^\star(\mathbf{p})$ \\
    \midrule
    Negative entropy & $\langle \mathbf{p}, \log \mathbf{p} - 1 \rangle$ & $\log(\mathbf{p})$ & $\exp(\mathbf{p})$ \\
    $\alpha$ norm & $\frac{1}{\alpha} \| \mathbf{p} \|_{\alpha}^{\alpha}$ & $\vp^{\odot (\alpha-1)}$ & $\frac{1}{\alpha} [\vp]_{+}^{\odot 1 / (\alpha-1)}$ \\
    $\alpha$ norm with sparsity constraint & $\frac{1}{\alpha} \| \mathbf{p} \|_{\alpha}^{\alpha} + \delta_{k}(\vp)$ & & $\frac{1}{\alpha} \sum_{i \in \integ{k}} [p_i]_{+}^{1 / (\alpha-1)}$ \\
    \bottomrule
  \end{tabular}
  \vspace{0.5cm}
\end{table}


% \begin{proposition}\label{prop:pcot}
%   Let $(\bm{a}, \bm{b}, \xi)$ be such that $\Pi(\bm{a}, \bm{b}) \cap \mathcal{B}_\psi(\xi)$ has an interior point and let $\mP^\star$ solve
%   \begin{align}\label{eq:pcOT}
%     \min_{\mP \in \Pi(\bm{a}, \bm{b})} \: \langle \mP, \C \rangle \quad \text{s.t.} \quad  \mP \in \mathcal{B}_\psi(\xi) \:.
%     \tag{OTARI-s}
% \end{align}
% Let $\bm{\varepsilon}^\star$ be the optimal dual variable associated with the constraint $\mP \in \mathcal{B}_\psi(\xi)$.
% If $\bm{\varepsilon}^\star \bm{>} \bm{0}$, then it holds $\mP^\star = \operatorname{Proj}^{D_\psi}_{\Pi(\bm{a}, \bm{b}) \cap \mathcal{B}_\psi(\xi)}(\mK_\sigma)$ for any $0 < \sigma \leq \min_i{\varepsilon_i^\star}$. Moreover it holds $\mP^\star = \nabla \psi^* \left(\diag(\bm{\varepsilon}^\star)^{-1} (\mC - \bm{\lambda}^\star \oplus \bm{\mu}^\star) \right)$ where $(\bm{\lambda}^\star, \bm{\mu}^\star, \bm{\varepsilon}^\star)$ solve the following dual
% \begin{align}
%   \max_{\bm{\lambda}, \bm{\mu}, \bm{\varepsilon} \bm{>} \bm{0}} \: \langle \bm{\lambda}, \bm{a} \rangle + \langle \bm{\mu}, \bm{b} \rangle + \left\langle \bm{\varepsilon}, \psi^*\left(\diag(\bm{\varepsilon})^{-1} (\mC - \bm{\lambda} \oplus \bm{\mu}) \right) - \psi(\mathbf{e}_\xi) \bm{1}  \right\rangle \:.
%   \tag{Dual-OTARI-s}
% \end{align}
% \end{proposition}

\begin{figure*}[t]
  \begin{center}
  \centerline{\includegraphics[width=\columnwidth]{chapters/OTARI/figures/visu_entropic_OTARI.pdf}}
  \caption{Entropic OT plans ($\xi=5$) with global constraint, pointwise constraints on sources and then on targets. The three plans have the same global entropy. The color of each source (resp. target) point shows the perplexity (exponential of entropy) of the associated row (resp. column) of the OT plan.}
  \label{fig:entropic_ot_plans}
  \end{center}
  % \vspace{-0.5cm}
\end{figure*}

% The doubly constrained problem, referred to as \eqref{eq:dpcOT}, is as follows for two perplexity parameters $\xi^{a}$ and $\xi^{b}$
% \begin{align}\label{eq:dpcOT}
%     \min_{\mP \in \Pi(\bm{a}, \bm{b})} \: \langle \mP, \mC \rangle \quad \text{s.t.} \quad  \mP \in \mathcal{B}_\psi(\xi^{a}) \quad \text{and} \quad \mP^\top \in \mathcal{B}_\psi(\xi^{b})
%     \tag{OTARI-d}
%   \end{align}
% and can also be seen as a $\psi$-Bregman projection through a trivial extension of \cref{prop:pcot}. \hug{à reprendre}

\begin{wrapfigure}[18]{L}{0.52\textwidth}
  \begin{minipage}{0.52\textwidth}
\begin{algorithm}[H]
  \caption{\textit{Dykstra} for solving (OTARI-d)}
  \label{algo:Dykstra_pcot}
  \begin{algorithmic}[1]
      \STATE {\textbf{Input}: $\mC$, $\psi(\cdot)$, $\xi^{a}$, $\xi^{b}$, $\varepsilon$}, $\bm{a}$, $\bm{b}$ \\
      % \STATE $t \leftarrow 0$ \\
      \STATE $\left(\mP_b, \bm{\Xi}, \bm{\Theta} \right) \leftarrow \left(\nabla \psi^{\star}(-\mC / \varepsilon), \mathbf{0}, \mathbf{0} \right)$ \\
      \WHILE{not converged}
          % \STATE $t \leftarrow t+1$ \\
          \STATE $\mP_a \leftarrow \operatorname{Proj}^{D_\psi}_{\Pi(\bm{a})}(\mP_b)$ 
          \\
          \STATE $\overline{\mP}_{a} \leftarrow \operatorname{Proj}^{D_\psi}_{\mathcal{B}_{\psi}(\xi^{a})}\circ \nabla \psi^*(\nabla \psi(\mP_a) + \bm{\Xi})$ 
          \\
          \STATE $\bm{\Xi} \leftarrow \bm{\Xi} + \nabla \psi(\mP_{a}) - \nabla \psi(\overline{\mP}_{a})$
          \\
          \STATE $\mP^\top_b \leftarrow \operatorname{Proj}^{D_\psi}_{\Pi(\bm{b})}(\overline{\mP}_{a}^\top)$ 
          \\
          \STATE $\overline{\mP}_{b}^\top \leftarrow \operatorname{Proj}^{D_\psi}_{\mathcal{B}_{\psi}(\xi^{b})}\circ \nabla \psi^*((\nabla \psi(\mP_b) + \bm{\Theta})^\top)$ 
          \\
          \STATE $\bm{\Theta} \leftarrow \bm{\Theta} + \nabla \psi(\mP_{b}) - \nabla \psi(\overline{\mP}_{b})$
      \ENDWHILE  
      \STATE {\bfseries Output: $\overline{\mP}_{b}$}
\end{algorithmic}
\end{algorithm}
\end{minipage}
\end{wrapfigure}

According to \cref{prop:pcot}, one can solve \eqref{eq:pcOT} using either alternating projections or dual ascent. 
% Note that upon convergence, dual ascent outputs the exact solution of \eqref{eq:pcOT} as it provides a set of variables $(\mP^\star, \bm{\lambda}^\star, \bm{\mu}^\star, \bm{\varepsilon}^\star)$ satisfying its KKT conditions. 
When $\bm{\varepsilon}^\star \bm{>} \bm{0}$, meaning that all constraints are active \ie $\forall i, \: \psi(\mP^\star_{i:}) = \psi(\mathbf{p}_{\xi})$, dual ascent is usually faster. However, if $\bm{\varepsilon}^\star$ has null components, one can still rely on $\operatorname{Proj}^{D_\psi}_{\Pi(\bm{a}, \bm{b}) \cap \mathcal{B}_\psi(\xi)}(\mK_\varepsilon)$ to provide an approximate solution as alternating Bregman projections are always guaranteed to converge.

Note that we can impose the pointwise constraint equivalently on the rows or the columns of the OT plan. Hence (OTARI-t) can be defined by imposing the constraint on the target samples \textit{i.e.}\ $\mP^\top \in \mathcal{B}_\psi(\xi)$.
We also propose a doubly constrained formulation called (OTARI-d) that consists of projecting $\mK_\sigma$ onto the nonempty set $\mathcal{B}_\psi(\xi^{a}) \cap \mathcal{B}^\top_\psi(\xi^b)$ 
% where $\xi^{a}$ and $\xi^{b}$ are chosen such that this intersection is nonempty 
where we defined $\mathcal{B}^\top_\psi(\xi) = \{ \mP^\top \in \mathcal{B}_\psi(\xi)\}$ thus ensuring sufficient smoothing for both rows and columns.
Such projection can be computed using alternating Bregman projections, whose convergence has been well-studied \cite{censor1998dykstra, benamou2015iterative}.
As we generally do not have access to a closed form for the projection onto the transport polytope $\Pi(\bm{a}, \bm{b})$, it is common to alternate projection onto $\Pi(\bm{a})$ and $\Pi(\bm{b})$ separately (see \eg the seminal Sinkhorn algorithm \cite{cuturi2013sinkhorn}).
We extend this scheme by adding projection steps into the pointwise constraints $\mathcal{B}_\psi(\xi)$ for both $\mP$ and $\mP^\top$. As this set is not affine, one needs to resort to the Dykstra procedure \cite{dykstra1983algorithm} that can be applied to a broad class of Bregman divergences \cite{bauschke2000dykstras}, as shown in \cref{algo:Dykstra_pcot}.
% It consists of adding corrective terms to the non-affine projection. 
In \cref{sec:proof_projs}, we provide the form of the projections for $\psi_{\KL}$ and $\psi_2$. A key benefit of decoupling both row and column constraints is that projection onto $\mathcal{B}_{\psi}(\xi)$ exhibits a simple structure where the rows can be decoupled into independent subproblems.
% In the latter, $\psi^*$ is the Fenchel conjugate of $\psi$.