% !TeX root = ../workshop_paper.tex

\section{Introduction}

Optimal transport (OT) is a well-established framework to compare probability distributions with numerous applications in machine learning \citep{arjovsky2017wasserstein, ozair2019wasserstein, peyre2019computational}.
Discrete OT seeks a transportation plan that minimizes the total transportation cost between samples from the source and target distributions.
In the absence of regularization, this optimal OT plan is inherently sparse. 
Regularizing OT with a strictly convex term is a widely adopted practical approach, leading to reduced numerical complexity and more diffuse OT plans \citep{peyre2019computational}.
As an illustration, the prominent entropic regularization \citep{cuturi2013sinkhorn} leads to a dense plan.
In some applications, the smoothing effect induced by the regularization has a primary importance on its own. A key example is the construction of doubly stochastic affinity matrices for clustering and dimensionality reduction \citep{landa2021doubly,Zass}, where smoothing enables connecting to neighbor points.
Another is domain adaptation \citep{courty2017joint} where smoothed OT often results in enhanced performance when compared to non-regularized ones (see for instance \cref{tab:da_exps}). Many OT regularization schemes on the primal formulation impose a scalar constraint on the global transport plan.
% \titouan{bon c'est pas si vrai il y a plein de manière de pénaliser: dual semi-dual, primal sur les colonnes etc... j'ai donc un peu changé}.
Consequently, central data points tend to exhibit a denser (or more diffuse) transport plan compared to extreme (or outlier) data points, for which diffusion is more costly. As a result, the latter points receive limited benefits from the smoothing effect introduced by the regularizer as shown in the left side of \cref{fig:entropic_ot_plans}. This partly explains OT's significant sensitivity to outliers in many applications \citep{mukherjee2021outlier, pmlr-v202-chuang23a}. 
% \titouan{la je pense qu'il faut parler de toute la littérature ‘‘outlier robust OT'' par exemple justin solomon mais il y en a plein d'autres}. 
To remedy this, one needs to constrain the transport plan in a pointwise manner.
Note that this has recently been explored for constructing affinity matrices \citep{van2023snekhorn} (\ie symmetric OT setting) leading to enhanced noise robustness and clustering abilities.

\paragraph{Contributions.} In this work, we develop a new formulation of OT, called OT with Adaptive Regularization (OTARI), allowing to control, for any strictly convex function $\psi$, the value of $\psi$ on each row and/or column of the OT plan. 
We then show the advantages of OTARI over usual regularized OT on domain adaptation tasks, focusing particularly on the negative entropy and the $\ell_2$ norm respectively associated with entropic \citep{cuturi2013sinkhorn} and quadratic \citep{blondel2018smooth} optimal transport.