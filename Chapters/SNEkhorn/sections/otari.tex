% !TeX root = ../workshop_paper.tex

\section{Optimal Transport with Adaptive Regularization}\label{sec:OTARI}

\subsection{Motivations}

Optimal transport (OT) is a well-established framework to compare probability distributions with numerous applications in machine learning \citep{arjovsky2017wasserstein, ozair2019wasserstein, peyre2019computational}.
Discrete OT seeks a transportation plan that minimizes the total transportation cost between samples from the source and target distributions.
In the absence of regularization, this optimal OT plan is inherently sparse. 
Regularizing OT with a strictly convex term is a widely adopted practical approach, leading to reduced numerical complexity and more diffuse OT plans \citep{peyre2019computational}.
As an illustration, the prominent entropic regularization \citep{cuturi2013sinkhorn} leads to a dense plan.
In some applications, the smoothing effect induced by the regularization has a primary importance on its own. A key example is the construction of doubly stochastic affinity matrices for clustering and dimensionality reduction \citep{landa2021doubly,Zass}, where smoothing enables connecting to neighbor points.
Another is domain adaptation \citep{courty2017joint} where smoothed OT often results in enhanced performance when compared to non-regularized ones (see for instance \cref{tab:da_exps}). Many OT regularization schemes on the primal formulation impose a scalar constraint on the global transport plan.
% \titouan{bon c'est pas si vrai il y a plein de manière de pénaliser: dual semi-dual, primal sur les colonnes etc... j'ai donc un peu changé}.
Consequently, central data points tend to exhibit a denser (or more diffuse) transport plan compared to extreme (or outlier) data points, for which diffusion is more costly. As a result, the latter points receive limited benefits from the smoothing effect introduced by the regularizer as shown in the left side of \cref{fig:entropic_ot_plans}. This partly explains OT's significant sensitivity to outliers in many applications \citep{mukherjee2021outlier, pmlr-v202-chuang23a}. 
% \titouan{la je pense qu'il faut parler de toute la littérature ‘‘outlier robust OT'' par exemple justin solomon mais il y en a plein d'autres}. 
To remedy this, one needs to constrain the transport plan in a pointwise manner.
Note that this has recently been explored for constructing affinity matrices \citep{van2023snekhorn} (\ie symmetric OT setting) leading to enhanced noise robustness and clustering abilities.

\paragraph{Contributions.} In this work, we develop a new formulation of OT, called OT with Adaptive Regularization (OTARI), allowing to control, for any strictly convex function $\psi$, the value of $\psi$ on each row and/or column of the OT plan. 
We then show the advantages of OTARI over usual regularized OT on domain adaptation tasks, focusing particularly on the negative entropy and the $\ell_2$ norm respectively associated with entropic \citep{cuturi2013sinkhorn} and quadratic \citep{blondel2018smooth} optimal transport.


\subsection{Formulation of the Problem}\label{sec:pwot}

In this section, we present a new formulation of OT that imposes constraints on each row of the OT plan.
We begin by introducing the set of matrices with \emph{point-wise} constraints. To set the upper bound, we rely on the perplexity parameter $\xi$ \citep{van2023snekhorn} that can be interpreted as the number of effective neighbors for each point. Concretely, we define $\mathbf{e}_{\xi} = \frac{1}{\xi}(\ind_{i \leq \xi})_{i}$ the probability vector of $\xi$ uniform arms and
\begin{align}
  \mathcal{B}_\psi(\xi) &\coloneqq \{\mP \bm{\geq} \bm{0} \: \text{s.t.}\ \: \forall i, \: \psi(\mP_{i:}) \leq \psi(\mathbf{e}_{\xi}) \} \:.
\end{align}
Note that in particular $\psi_{\KL}(\mathbf{e}_{\xi}) = - (\log \xi + 1)$ and $\psi_2(\mathbf{e}_{\xi}) = 1/\xi$.
% In contrast to the constraint utilized in standard strictly convex OT \eqref{eq:scOT}, $\mathcal{B}_\psi(\xi)$ imposes a constraint on \emph{each row} of the transport plan $\mP$.
We now define Optimal Transport with Adaptive Regularization (OTARI) as the generalization of \eqref{eq:cot} to the case where the strictly convex constraint is given by $\mathcal{B}_\psi(\xi)$. Similarly to \Cref{prop:cot}, we can frame OTARI as a $\psi$-Bregman projection of $\mK_\sigma = \nabla \psi^*(-\mC / \sigma)$ or solve it using dual ascent.

\paragraph{OTARI problem.}
  Let $(\bm{a}, \bm{b}, \xi)$ be such that $\gU(\bm{a}, \bm{b}) \cap \mathcal{B}_\psi(\xi)$ has an interior point.
  We consider the following problem:
  \begin{align}\label{eq:pcOT}
    \min_{\mP \in \gU(\bm{a}, \bm{b})} \: \langle \mP, \mC \rangle \quad \text{s.t.} \quad  \mP \in \mathcal{B}_\psi(\xi) \:.
    \tag{OTARI-s}
\end{align}

\paragraph{Mirror descent.}
We propose to solve the above using a mirror descent algorithm that consists in the following update of the primal variable $\mP$:
\begin{align}
  \mP_{t+1} &= \argmin_{\mP \in \gU(\bm{a}, \bm{b}) \cap \mathcal{B}_\psi(\xi)} \: \scalar{\mP}{\mC} + \frac{1}{\gamma_t} D_{\psi}(\mP, \mP_t) \\
  &= \argmin_{\mP \in \gU(\bm{a}, \bm{b}) \cap \mathcal{B}_\psi(\xi)} \: \psi(\mP) - \scalar{\mP}{\nabla \psi(\mP_t) - \gamma_t \mC} \\
  &= \operatorname{Proj}^{D_\psi}_{\gU(\bm{a}, \bm{b}) \cap \mathcal{B}_\psi(\xi)}(\nabla \psi^*(\nabla \psi(\mP_t) - \gamma_t \mC)) \:.
\end{align}

% \begin{table}[h]
%   \centering
%   \caption{Table caption}
%   \label{tab:example}
%   \begin{tabular}{lccc}
%     \toprule
%     & $\psi(\mathbf{p})$ & $\nabla \psi(\mathbf{p})$ & $\nabla \psi^\star(\mathbf{p})$ \\
%     \midrule
%     Negative entropy & $\langle \mathbf{p}, \log \mathbf{p} - 1 \rangle$ & $\log(\mathbf{p})$ & $\exp(\mathbf{p})$ \\
%     $\alpha$ norm & $\frac{1}{\alpha} \| \mathbf{p} \|_{\alpha}^{\alpha}$ & $\vp^{\odot (\alpha-1)}$ & $\frac{1}{\alpha} [\vp]_{+}^{\odot 1 / (\alpha-1)}$ \\
%     $\alpha$ norm with sparsity constraint & $\frac{1}{\alpha} \| \mathbf{p} \|_{\alpha}^{\alpha} + \delta_{k}(\vp)$ & & $\frac{1}{\alpha} \sum_{i \in \integ{k}} [p_i]_{+}^{1 / (\alpha-1)}$ \\
%     \bottomrule
%   \end{tabular}
%   \vspace{0.5cm}
% \end{table}


\begin{proposition}\label{prop:pcot}
  Let $(\bm{a}, \bm{b}, \xi)$ be such that $\gU(\bm{a}, \bm{b}) \cap \mathcal{B}_\psi(\xi)$ has an interior point and let $\mP^\star$ solve
  \begin{align}\label{eq:pcOT}
    \min_{\mP \in \gU(\bm{a}, \bm{b})} \: \langle \mP, \C \rangle \quad \text{s.t.} \quad  \mP \in \mathcal{B}_\psi(\xi) \:.
    \tag{OTARI-s}
\end{align}
Let $\bm{\varepsilon}^\star$ be the optimal dual variable associated with the constraint $\mP \in \mathcal{B}_\psi(\xi)$.
If $\bm{\varepsilon}^\star \bm{>} \bm{0}$, then it holds $\mP^\star = \operatorname{Proj}^{D_\psi}_{\gU(\bm{a}, \bm{b}) \cap \mathcal{B}_\psi(\xi)}(\mK_\sigma)$ for any $0 < \sigma \leq \min_i{\varepsilon_i^\star}$. Moreover it holds $\mP^\star = \nabla \psi^* \left(\diag(\bm{\varepsilon}^\star)^{-1} (\mC - \bm{\lambda}^\star \oplus \bm{\mu}^\star) \right)$ where $(\bm{\lambda}^\star, \bm{\mu}^\star, \bm{\varepsilon}^\star)$ solve the following dual
\begin{align}
  \max_{\bm{\lambda}, \bm{\mu}, \bm{\varepsilon} \bm{>} \bm{0}} \: \langle \bm{\lambda}, \bm{a} \rangle + \langle \bm{\mu}, \bm{b} \rangle + \left\langle \bm{\varepsilon}, \psi^*\left(\diag(\bm{\varepsilon})^{-1} (\mC - \bm{\lambda} \oplus \bm{\mu}) \right) - \psi(\mathbf{e}_\xi) \bm{1}  \right\rangle \:.
  \tag{Dual-OTARI-s}
\end{align}
\end{proposition}

\begin{figure*}[t]
  \begin{center}
  \centerline{\includegraphics[width=\columnwidth]{figures/OTARI/visu_entropic_OTARI.pdf}}
  \caption{Entropic OT plans ($\xi=5$) with global constraint, pointwise constraints on sources and then on targets. The three plans have the same global entropy. The color of each source (resp. target) point shows the perplexity (exponential of entropy) of the associated row (resp. column) of the OT plan.}
  \label{fig:entropic_ot_plans}
  \end{center}
  % \vspace{-0.5cm}
\end{figure*}

% The doubly constrained problem, referred to as \eqref{eq:dpcOT}, is as follows for two perplexity parameters $\xi^{a}$ and $\xi^{b}$
% \begin{align}\label{eq:dpcOT}
%     \min_{\mP \in \gU(\bm{a}, \bm{b})} \: \langle \mP, \mC \rangle \quad \text{s.t.} \quad  \mP \in \mathcal{B}_\psi(\xi^{a}) \quad \text{and} \quad \mP^\top \in \mathcal{B}_\psi(\xi^{b})
%     \tag{OTARI-d}
%   \end{align}
% and can also be seen as a $\psi$-Bregman projection through a trivial extension of \cref{prop:pcot}. \hug{à reprendre}

\begin{wrapfigure}[18]{L}{0.52\textwidth}
  \begin{minipage}{0.52\textwidth}
\begin{algorithm}[H]
  \caption{\textit{Dykstra} for solving (OTARI-d)}
  \label{algo:Dykstra_pcot}
  \begin{algorithmic}[1]
      \STATE {\textbf{Input}: $\mC$, $\psi(\cdot)$, $\xi^{a}$, $\xi^{b}$, $\varepsilon$}, $\bm{a}$, $\bm{b}$ \\
      % \STATE $t \leftarrow 0$ \\
      \STATE $\left(\mP_b, \bm{\Xi}, \bm{\Theta} \right) \leftarrow \left(\nabla \psi^{\star}(-\mC / \varepsilon), \mathbf{0}, \mathbf{0} \right)$ \\
      \WHILE{not converged}
          % \STATE $t \leftarrow t+1$ \\
          \STATE $\mP_a \leftarrow \operatorname{Proj}^{D_\psi}_{\gU(\bm{a})}(\mP_b)$ 
          \\
          \STATE $\overline{\mP}_{a} \leftarrow \operatorname{Proj}^{D_\psi}_{\mathcal{B}_{\psi}(\xi^{a})}\circ \nabla \psi^*(\nabla \psi(\mP_a) + \bm{\Xi})$ 
          \\
          \STATE $\bm{\Xi} \leftarrow \bm{\Xi} + \nabla \psi(\mP_{a}) - \nabla \psi(\overline{\mP}_{a})$
          \\
          \STATE $\mP^\top_b \leftarrow \operatorname{Proj}^{D_\psi}_{\Pi(\bm{b})}(\overline{\mP}_{a}^\top)$ 
          \\
          \STATE $\overline{\mP}_{b}^\top \leftarrow \operatorname{Proj}^{D_\psi}_{\mathcal{B}_{\psi}(\xi^{b})}\circ \nabla \psi^*((\nabla \psi(\mP_b) + \bm{\Theta})^\top)$ 
          \\
          \STATE $\bm{\Theta} \leftarrow \bm{\Theta} + \nabla \psi(\mP_{b}) - \nabla \psi(\overline{\mP}_{b})$
      \ENDWHILE  
      \STATE {\bfseries Output: $\overline{\mP}_{b}$}
\end{algorithmic}
\end{algorithm}
\end{minipage}
\end{wrapfigure}

According to \cref{prop:pcot}, one can solve \eqref{eq:pcOT} using either alternating projections or dual ascent. 
% Note that upon convergence, dual ascent outputs the exact solution of \eqref{eq:pcOT} as it provides a set of variables $(\mP^\star, \bm{\lambda}^\star, \bm{\mu}^\star, \bm{\varepsilon}^\star)$ satisfying its KKT conditions. 
When $\bm{\varepsilon}^\star \bm{>} \bm{0}$, meaning that all constraints are active \ie $\forall i, \: \psi(\mP^\star_{i:}) = \psi(\mathbf{p}_{\xi})$, dual ascent is usually faster. However, if $\bm{\varepsilon}^\star$ has null components, one can still rely on $\operatorname{Proj}^{D_\psi}_{\gU(\bm{a}, \bm{b}) \cap \mathcal{B}_\psi(\xi)}(\mK_\varepsilon)$ to provide an approximate solution as alternating Bregman projections are always guaranteed to converge.

Note that we can impose the pointwise constraint equivalently on the rows or the columns of the OT plan. Hence (OTARI-t) can be defined by imposing the constraint on the target samples \textit{i.e.}\ $\mP^\top \in \mathcal{B}_\psi(\xi)$.
We also propose a doubly constrained formulation called (OTARI-d) that consists of projecting $\mK_\sigma$ onto the nonempty set $\mathcal{B}_\psi(\xi^{a}) \cap \mathcal{B}^\top_\psi(\xi^b)$ 
% where $\xi^{a}$ and $\xi^{b}$ are chosen such that this intersection is nonempty 
where we defined $\mathcal{B}^\top_\psi(\xi) = \{ \mP^\top \in \mathcal{B}_\psi(\xi)\}$ thus ensuring sufficient smoothing for both rows and columns.
Such projection can be computed using alternating Bregman projections, whose convergence has been well-studied \citep{censor1998dykstra, benamou2015iterative}.
As we generally do not have access to a closed form for the projection onto the transport polytope $\gU(\bm{a}, \bm{b})$, it is common to alternate projection onto $\gU(\bm{a})$ and $\Pi(\bm{b})$ separately (see \eg the seminal Sinkhorn algorithm \citep{cuturi2013sinkhorn}).
We extend this scheme by adding projection steps into the pointwise constraints $\mathcal{B}_\psi(\xi)$ for both $\mP$ and $\mP^\top$. As this set is not affine, one needs to resort to the Dykstra procedure \citep{dykstra1983algorithm} that can be applied to a broad class of Bregman divergences \citep{bauschke2000dykstras}, as shown in \cref{algo:Dykstra_pcot}.
% It consists of adding corrective terms to the non-affine projection. 
In \cref{sec:proof_projs}, we provide the form of the projections for $\psi_{\KL}$ and $\psi_2$. A key benefit of decoupling both row and column constraints is that projection onto $\mathcal{B}_{\psi}(\xi)$ exhibits a simple structure where the rows can be decoupled into independent subproblems.
% In the latter, $\psi^*$ is the Fenchel conjugate of $\psi$.


\subsection{Application to Domain Adaptation}

\begin{figure}[h]
  \begin{center}
  \centerline{\includegraphics[width=\columnwidth]{figures/OTARI/visu_DA.pdf}}
  \caption{Toy domain adaptation scenario with entropic OT plans ($\xi=10$) with various constraints. The size of the point is proportional to the associated entropy. When using Sinkhorn, barycentric mapping match outliers points since the OT plan is less diffuse for these points. In turn, using pointwise constraints concentrate the 
   mapped points in high-density regions, thus giving more robust estimates for the mappings onto the target domain.}
  \label{fig:visu_DA}
  \end{center}
\end{figure}

\begin{table}[h]
  % \begin{wraptable}[7]{L}{8.5cm}
    % \vspace*{-0.4cm}
    \centering
    % Note that unregularized OT does not depend on $\xi$. }
    \begin{small}
    \begin{tabular}{lc@{\hskip 0.1in}c@{\hskip 0.1in}c@{\hskip 0.1in}c@{\hskip 0.1in}c@{\hskip 0.1in}c}
    \toprule[1.5pt]
    & OT& EOT & EOTARI-s & EOTARI-t & EOTARI-d \\
    \midrule
    MNIST $\to$ USPS ($\xi=30$) & $53.1(5.4)$ & $64.2(2.8)$ & $65.0(5.3)$ & $66.4(3.5)$ & $\mathbf{67.4(2.9)}$ \\
    MNIST $\to$ USPS ($\xi=300$) & $53.1(5.4)$ & $68.8(3.1)$ & $70.8(4.2)$ & $70.2(3.4)$ & $\mathbf{72.6(5.1)}$ \\
  
    USPS $\to$ MNIST ($\xi=30$) & $59.1(4.9)$ & $60.8(5.4)$ & $61.6(4.4)$ & $\mathbf{62.6(3.0)}$ & $61.0(4.7)$ \\
    USPS $\to$ MNIST ($\xi=300$) & $59.1(4.9)$ & $59.8(1.6)$ & $61.0(2.3)$ & $\mathbf{61.6(3.0)}$ & $58.8(2.3)$ \\
    \midrule
    \midrule
    & OT & QOT & QOTARI-s & QOTARI-t & QOTARI-d \\
    \midrule
    MNIST $\to$ USPS ($\xi=30$) & $53.1(5.4)$ & $68.3(3.9)$ & $68.3(3.6)$ & $\mathbf{69.3(4.7)}$ & $68.1(4.6)$ \\
    MNIST $\to$ USPS ($\xi=300$) & $53.1(5.4)$ & $60.7(1.5)$ & $\mathbf{67.0(2.4)}$ & $65.5(2.3)$ & $65.8(2.5)$ \\
    USPS $\to$ MNIST ($\xi=30$) & $59.1(4.9)$ & $60.4(3.5)$ & $\mathbf{62.8(3.7)}$ & $59.6(2.7)$ & $61.6(3.1)$ \\
    USPS $\to$ MNIST ($\xi=300$) & $59.1(4.9)$ & $59.2(3.4)$ & $60.1(3.0)$ & $\mathbf{62.0(3.7)}$ & $61.5(3.8)$ \\
    \bottomrule[1.5pt]
    \end{tabular}
    \end{small}
    \vspace{0.4cm}
    \caption{Domain adaptation 1-NN classification scores for OT (unregularized), EOT (entropic), EOTARI (entropic OTARI), QOT (quadratic), QOTARI (quadratic OTARI) for $\xi=30$ and $\xi=300$.}
    \label{tab:da_exps}
    \vspace{0.1cm}
  \end{table}

In this section, we evaluate OTARI on a domain adaptation task where the goal is to transport labeled data points to a target domain where a classifier is trained. Mapping onto the target domain is performed through a barycentric mapping of the form: for any $i \in \integ{N_S}$, $\hat{\bm{x}}_i = \frac{1}{a_i} \sum_j T_{ij} \bm{x}_j^T$.
Looking at \cref{fig:visu_DA}, one can notice that using OTARI for domain adaptation yields a mapping that is concentrated in high-density (thus more faithful) regions of the target domain. On the opposite, when using globally constrained OT (left side of \cref{fig:visu_DA}), the barycentric mapping associated with an outlier is concentrated on the outlier's position. For the experiments, we take $\C$ as the squared Euclidean distance computed from raw images of the handwritten digit classification benchmark MNIST-USPS. 
Following the standard practice in OT-based domain adaptation \cite{flamary2016optimal}, we map the source to the target samples and then train a 1-NN classifier on the barycentric mappings with source labels.
We compute the outcomes across 10 independent trials. In each of these experiments, the target data is partitioned into a 90\% training and 10\% testing split, with OT barycentric mappings and 1-NN classifiers exclusively applied to the training set. Mean scores and standard deviations are displayed in \cref{tab:da_exps}. The latter shows that adaptive regularization consistently outperforms global regularization (set such that $\sum_i \psi(\mP^\star_{i:}) = n \psi(\mathbf{e}_{\xi})$ where $\mP^\star$ solves \ref{eq:cot} for a fair comparison) with significant performance gains in some settings.

\paragraph{Possible extensions.}
Among possible extensions of \Cref{eq:pcOT}, one could also extend OTARI to continuous distributions and apply it to OT mapping
estimation \cite{pooladian2021entropic}. Other interesting directions include
investigating optimization algorithms that can avoid quadratic memory complexity.
