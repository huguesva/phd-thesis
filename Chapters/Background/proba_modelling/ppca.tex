\section{The Latent Variable Perspective}\label{sec:dr_proba_modelling}

After presenting the fundamental concepts of dimensionality reduction (DR) as the matching of two affinity matrices $\mA_{\mX}$ and $\mA_{\mZ}$ in \Cref{sec:background_dr}, we now offer a fresh perspective through the lens of probabilistic modeling.

In this perspective, the fundamental assumption is to view the data matrix $\mX$ as the result of a generative process based on an underlying latent low-dimensional variable $\mZ$. This interpretation justifies the use of the term \emph{latent space}, which we use interchangeably with \emph{embedding space} throughout this thesis.

This viewpoint is especially relevant for linear DR methods, such as PCA and its probabilistic variant, Probabilistic PCA (PPCA) \citep{tipping1999probabilistic}. It also prompts the question of how to extend this framework to more general DR formulations, including neighbor embedding approaches. This sets the stage for the material introduced in \Cref{chapter:GraphCoupling}.

\subsection{From Probabilistic PCA ...}\label{sec:ppca}

In probabilistic PCA (\emph{PPCA}), we make the key modelling assumption that the data $\mX$ is generated through a linear transformation of the latent variables $\mZ$, to which Gaussian noise is added. This model is thus defined by the following generative process:
\begin{align}
    \mX = \mW \mZ + \mE,
\end{align}
where $\mX \in \mathbb{R}^{N \times p}$ is the data matrix, $\mZ \in \mathbb{R}^{N \times d}$ is the embedding matrix, $\mW \in \mathbb{R}^{p \times d}$ is the linear transformation matrix, and $\mE \in \mathbb{R}^{N \times p}$ is the noise matrix. The noise is assumed to consist of independent and identically distributed (i.i.d.) Gaussian variables, i.e., $\mE = (\ve_1, \ldots, \ve_N)^\top$, where each $\ve_i$ follows a Gaussian distribution $\mathcal{N}(0, \sigma^2 \mI_p)$, with $\mI_p$ being the $p \times p$ identity matrix and $\sigma^2$ the variance of the noise.

Importantly, the latent variables are also assumed to be \emph{iid} Gaussian distributed \ie for any $i$, $\vz_i \sim \mathcal{N}(0, \mI_d)$. With this in place, the model is fully specified by the parameters $\theta = (\mW, \sigma^2)$.

Note that we can rewrite the above linear relation as
\begin{align}
    \forall i, \quad \vx_i = \mW \vz_i + \ve_i \:.
\end{align}
In this quite simple model, the $\vx_i$'s are independent as the samples in $\mZ$ and $\mE$ are assumed to be independent.

\paragraph{Parameter estimation.} A standard approach to fitting latent variable models consists in marginalizing over the latent variables and optimizing the parameters using \emph{maximum likelihood estimation} (MLE).
Since, $\ve_i \sim \mathcal{N}(0, \sigma^2 \mI_p)$ for any $i$, one has 
\begin{align}
    \forall i, \quad \vx_i | \left( \vz_i, \mW, \sigma \right) \: \sim \mathcal{N}(\mW \vx_i, \sigma^2 \mI_p) \:.
\end{align}
Integrating over the latent variable $\vz_i \sim \mathcal{N}(0, \mI_d)$, one obtains the marginal distribution of $\vx_i$ as follows
\begin{align}
    \vx_i | \left( \mW, \sigma \right) \: \sim \mathcal{N}(0, \mW \mW^\top + \sigma^2 \mI_p) \:.
\end{align}

Since the $\vx_i$'s are \emph{iid}, the log marginal likelihood can be experessed as the sum of the log-likelihoods of the individual samples. With matrix notations, we are left with the following problem, where we simplified constant terms
\begin{align}
    \max_{\mW, \sigma} \: -N \log |\mC| - \mX \mC^{-1} \mX^\top \quad \text{where} \quad \mC = \mW \mW^\top + \sigma^2 \mI_p \:.
\end{align}
As shown in \citep{tipping1999probabilistic}, the above is solved for 

\begin{remark}
    later reason the the inter sample dependencies
\end{remark}

The negative log-likelihood is as follows:
\begin{align}
    \mathcal{L}(\theta) = \frac{1}{2\sigma^2} \| \mX - \mW \mZ \|_F^2 + \frac{Np}{2} \log(2\pi\sigma^2),
\end{align}
where $\| \cdot \|_F$ denotes the Frobenius norm.

\begin{remark}
    Probabilistic PCA as a cornerstone of generative modelling    

    \citep{tomczak2021latent}
\end{remark}

\subsection{... to its Bayesian Dual View}


\citep{lawrence2005probabilistic}


\begin{prob}{Construction of adaptive affinities}
    aa
\end{prob}