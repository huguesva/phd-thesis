\chapter{Introduction}\label{chap:intro}

\minitoc

\section{Building dense representations of the data}

Exploring and analyzing high-dimensional data is a core problem of data science. As presented in \Cref{c-intro}, it often requires constructing dense low-dimensional and interpretable representations of the data. 


Typically, this data is redundant and noisy, making it difficult to directly extract useful information from the raw data. Therefore, the challenge of finding a compact and meaningful representation of high-dimensional data becomes paramount.


One major objective of unsupervised learning~\citep{Hastie2009} is to provide interpretable and meaningful approximate representations of the data that best preserve its structure \ie the underlying geometric relationships between the data samples.
Similar in essence to Occam's principle frequently employed in supervised learning, the preference for unsupervised data representation often aligns with the pursuit of simplicity, interpretability or visualizability in the associated model.
These aspects are determinant in many real-world applications, such as cell biology
\citep{cantini2021benchmarking, ventre2023one}, where the interaction with domain experts is paramount for interpreting the results and extracting meaningful insights from the model. 

Ideally, the reduced representation should align with the intrinsic dimensionality of the data, which broadly refers to the minimum number of parameters needed to capture its essential characteristics.






\paragraph{The AI revolution is happening in latent space.}


\section{What is referred to as dimensionality reduction?}


When a ground distance (or similarity) metric is available on $\mathcal{X}$, learning data representations translates into solving a dimensionality reduction (DR) problem.


Dimensionality reduction (DR) is of central importance when dealing with high-dimensional data \citep{donoho2000high}. It mitigates the curse of dimensionality, allowing for greater statistical flexibility and less computational complexity. DR also enables visualization that can be of great practical interest for understanding and interpreting the structure of large datasets.
Most seminal approaches include Principal Component Analysis (PCA) \citep{pearson1901liii},  multidimensional scaling \citep{kruskal1978multidimensional} and more broadly kernel eigenmaps methods such as Isomap \citep{balasubramanian2002isomap}, Laplacian eigenmaps \citep{belkin2003laplacian} and diffusion maps \citep{coifman2006diffusion}. These methods share the definition of a pairwise similarity kernel that assigns a high value to close neighbors and the resolution of a spectral problem. They are well understood and unified in the kernel PCA framework \citep{ham2004kernel}.

In the past decade, the field has witnessed a major shift with the emergence of a new class of methods. They are also based on pairwise similarities but these are not converted into inner products. Instead, they define pairwise similarity functions in both input and latent spaces and optimize a cost between the two. Among such methods, the Stochastic Neighbor Embedding (SNE) algorithm \citep{NIPS2002SNE}, its heavy-tailed symmetrized version t-SNE \citep{maaten2008tSNE} or more recent approaches like LargeVis \citep{tang2016visualizing} and UMAP \citep{mcinnes2018umap} are arguably the most used in practice. These will be referred to as \textit{SNE-like} or \textit{neighbor embedding} methods in what follows. They are increasingly popular and now considered state-of-art techniques in many fields \citep{li2017application,kobak2019art,anders2018dissecting}. Their popularity is mainly due to their exceptional ability to preserve the local structure, \textit{i.e.}\ close points in the input space have close embeddings, as shown empirically \citep{wang2021understanding}. They also demonstrate impressive performances in identifying clusters \citep{arora2018analysis, linderman2019clustering}. However this is done at the expense of global structure, that these methods struggle in preserving \citep{wattenberg2016use, coenen2019understanding} \textit{i.e.}\ the relative large-scale distances between embedded points do not necessarily correspond to the original ones.

Given a dataset $\mathbf{X} \in \mathbb{R}^{n \times p}$ of $n$ samples in dimension $p$, most DR algorithms compute a representation of $\mathbf{X}$ in a lower-dimensional latent space $\mathbf{Z} \in \mathbb{R}^{n \times q}$ with $q \ll p$ that faithfully captures and represents pairwise dependencies between the samples (or rows) in $\X$. This is generally achieved by optimizing $\mathbf{Z}$ such that the corresponding affinity matrix matches another affinity matrix defined from $\X$. These affinities are constructed from a matrix $\mathbf{C} \in \R^{n \times n}$ that encodes a notion of ‘‘distance'' between the samples, \eg the squared Euclidean distance $C_{ij} = \|\mathbf{X}_{i:}-\mathbf{X}_{j:}\|_2^2$ or more generally any \emph{cost matrix} $\mathbf{C} \in \mathcal{D} := \{\Cb \in \mathbb{R}_+^{n \times n} : \Cb = \Cb^\top \text{and } C_{ij}=0 \iff i=j\}$. A commonly used option is the Gaussian affinity that is obtained by performing row-wise normalization of the kernel $\exp(-\Cb / \varepsilon)$, where $\varepsilon >0$ is the bandwidth parameter.




\paragraph{Two sides of the same coin.} As a matter of fact, methods from both families share many similitudes, %among which is
including the construction of a similarity graph between input samples. In clustering, many popular approaches design a reduced or coarsened version of the initial similarity graph while preserving some of its spectral properties~\citep{von2007tutorial, schaeffer2007graph}. 
%attributes related to the graph spectrum 
In DR, the goal is to solve the inverse problem of finding low-dimensional embeddings that generate a similarity graph close to the %initial
one computed from input data points \citep{ham2004kernel,hinton2002stochastic}.
%With so many similarities
Our work builds on these converging viewpoints and addresses the following question: \emph{can DR and clustering  be expressed in a common and unified framework ?}



\paragraph{Motivations behind latent variable models.} 
Consider a generative setting where the goal is to sample new data points from an existing dataset. The objective is to learn the underlying data distribution and generate samples from it. Latent variable models offer an intuitive and hierarchical approach to this problem \citep{bishop2006pattern}. For example, suppose we aim to generate images of flowers. We might begin by defining the basic structure: the stem, petals, and leaves. Then, we could refine the details, such as the number, shape, and color of the petals, as well as smaller elements like texture or background features (e.g., soil or sky) to produce a realistic final image. This step-by-step process reflects how we would naturally construct a drawing or image of a flower.


\section*{Outline of the thesis}

\paragraph{Chapter 2.}

\paragraph{Chatper 3.} We introduce a new probabilistic model for dimensionality reduction.

\paragraph{Chapter 4.} We propose a new algorithm for clustering.


\paragraph{Chapter 5.} 

\text


\begin{rem2}
	This is a test aa.
\end{rem2}




\subsection*{Acronyms.}
\begin{itemize}
	\item DR: Dimensionality Reduction.
	\item OT: Optimal Transport.
	\item GW: Gromov-Wasserstein.
	\item DS: Doubly Stochastic.
\end{itemize}