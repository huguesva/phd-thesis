\chapter{Introduction}\label{chap:intro}

\minitoc

\section{Representation Learning at the Core of Data Science}

When working with unlabeled data, practitioners often rely on unsupervised learning tools to gain insights that help them adapt subsequent analysis pipelines. Thus, building tools to explore and analyze high-dimensional data has become a core challenge in modern data science \citep{donoho2000high, rudin2022interpretable}. The primary difficulty lies in developing interpretable and meaningful approximate representations of the data that effectively reveal its structure, particularly the underlying geometric relationships between data samples.

These aspects are critical in many real-world applications, such as cell biology \citep{regev2017human, kobak2019art, becht2019dimensionality}, where collaboration with domain experts is essential for interpreting results, extracting meaningful insights from models, and refining experimental designs accordingly.

Importantly, real-world data is typically both redundant and noisy, so the models must account for this complexity and normalize the noise to capture the underlying meaningful structure. Ideally, the reduced representation should align with the intrinsic dimensionality of the data, which refers to the minimum number of parameters required to capture its essential characteristics. However, the definition of these essential characteristics depends on the specific downstream task being addressed, making the problem somewhat ill-defined.


\paragraph{On the key role of latent spaces.} 
Beyond simple data exploration, building meaningful representations significantly enhances the performance of machine learning models in general \citep{bengio2013representation}. This is because operating directly in input space is often suboptimal.

For instance, in the case of images, the distance between raw pixel values can be misleading, as even slight deformations, like rotations, can drastically change pixel values while the overall representation of the image remains the same. Therefore, it is preferable to work in a latent space where points are close in terms of conceptual similarity. Concretely, horses and zebras should be closer to each other than horses and goldfish.

Latent spaces enable models to capture higher-level abstractions, leading to improved generalization and greater learning efficiency. There are numerous examples of such models for tasks like planning \citep{hafner2019dream}, latent diffusion for image generation \citep{rombach2022high}, and joint embedding models for downstream tasks in image and video processing \citep{assran2023self, bardes2023v}.


\section{What is Dimensionality reduction in this Landscape}

Learning representations from data is a broad problem in machine learning that spans many families of methods. In this thesis, we focus specifically on the family often referred to as \emph{dimensionality reduction (DR)} methods. These methods address the particular case of representation learning when a \textbf{ground distance (or similarity) metric is available in the input space}. This ground metric can be leveraged as a form of supervision, guiding the model to reproduce a similar inter-sample dependency structure in the latent space (we detail this view in \Cref{sec:background_dr}). Methods that address such problems are therefore classified as DR methods.

In 2000, the machine learning community launched the research on manifold learning, which is to discover the
intrinsic structure of high dimensional data. 

Notably, defining a meaningful distance in the input space is not trivial for most data modalities. For example, in the case of images, data augmentations are often employed to either approximate a ground truth distance or circumvent the need for one \citep{chen2020simple}. However, other modalities are more conducive to providing a meaningful distance. This is evident, for instance, in single-cell genomics \citep{amir2013visne}, where variations in gene expression encode biologically significant information (see, \eg \Cref{fig:intro_fig}).


\begin{figure}[t]
    \centering
    \includegraphics[width=\linewidth]{figures/single_cell_readme.png}
    \caption{TSNE and LargeVis embeddings of single cell RNA-seq data from \citep{macosko2015highly} and \citep{zheng2017massively}. The colors represent different cell types. The embeddings are computed using the TorchDR toolbox \citep{vanassel2024torchdr}.
    }
    \label{fig:intro_fig}
\end{figure}



\paragraph{Classical approaches to dimensionality reduction.}
The most foundational methods involve linearly projecting the data onto a lower-dimensional space. These approaches, including Principal Component Analysis (PCA) \citep{pearson1901liii} and multidimensional scaling \citep{kruskal1978multidimensional}, are among the most well-established and widely studied techniques in dimensionality reduction. Their mechanisms can be extended by considering various types of similarity measures between data points in the input space. This leads to kernel-based methods such as Isomap \citep{balasubramanian2002isomap}, Laplacian eigenmaps \citep{belkin2003laplacian}, and diffusion maps \citep{coifman2006diffusion}. These methods share a common structure: they define a pairwise similarity kernel that assigns high values to nearby neighbors and solve a spectral problem. They are well understood and can be unified under the kernel PCA framework \citep{ham2004kernel}.

\paragraph{The revolution of neighbor embedding methods.}
In the past decade, the field has undergone a significant transformation with the rise of a new class of methods that have largely surpassed classical spectral approaches in the context of data visualization. 

Informally, the algorithm places all
points on the 2D plane, initially at random positions, and lets them
interact as if they were physical particles. The interaction is governed by two laws: first, all points are repelled from each other;
second, each point is attracted to its nearest neighbours

Prominent examples of such methods include the Stochastic Neighbor Embedding (SNE) algorithm \citep{NIPS2002SNE}, its heavy-tailed and symmetrized version t-SNE \citep{maaten2008tSNE}, and more recent approaches like LargeVis \citep{tang2016visualizing} and UMAP \citep{mcinnes2018umap}. These methods, referred to as \textit{SNE-like} or \textit{neighbor embedding} methods, have become increasingly popular and are now considered state-of-the-art in many fields \citep{li2017application,kobak2019art,anders2018dissecting}.
 
Their popularity stems from their remarkable ability to preserve local structure \ie, nearby points in the input space remain close in the embedding space, as shown empirically \citep{wang2021understanding}. Moreover, they excel at identifying clusters \citep{arora2018analysis, linderman2019clustering}, although this often comes at the expense of preserving global structure \citep{wattenberg2016use, coenen2019understanding}. Finally, these methods are highly scalable and particularly well-suited for GPU computation.


Although they are widely used in practice, there remains a significant gap in the understanding of neighbor embedding methods compared to well-known spectral algorithms. Addressing this gap is one of the primary objectives of this thesis.


\section{Outline of the thesis}

This thesis is organized as follows.

\paragraph{Chapter 2.}
We review all the necessary background material for the subsequent chapters. We begin with an overview of various dimensionality reduction (DR) methods through the lens of the affinity matching problem, where affinities are defined in both input and latent spaces, and the goal is to match them. We then provide two perspectives on DR methods: one based on latent variable models and the other from a distributional viewpoint using the framework of optimal transport. Extending these perspectives to neighbor embedding methods will be the primary focus of \Cref{chapter:GraphCoupling} and \Cref{chapter:DistR}. Finally, we raise the question of designing new affinity matrices that are more robust to noise heterogeneities in order to address real-world scenarios. This motivates the work presented in \Cref{chapter:SNEkhorn}.

\paragraph{Chatper 3.} 
Latent variable models offer a grounded and intuitive formulation of dimensionality reduction (DR). Their key advantage is that they enable extensions to various models and the adaptation of priors. However, current formulations of DR as latent variable models are restricted to linear DR methods, thus excluding popular approaches such as neighborhood embedding methods. 

In \Cref{chapter:GraphCoupling}, we introduce a novel latent variable model that broadly encompasses both linear and non-linear dimensionality reduction (DR) methods. This framework allows us to conceptualize these methods in a probabilistic manner, thus facilitating the incorporation of priors and providing a clearer understanding of the underlying modeling assumptions when applying these methods in practice.

\paragraph{Chapter 4.}  
Existing dimensionality reduction (DR) approaches primarily focus on computing a unique embedding for each input data, resulting in the same number of points in the latent space as in the input space. In parallel with this DR framework, clustering methods are often employed to group points into meaningful clusters that can be easily inspected. This approach can be understood as a way to control the \emph{granularity} of the embedding, as aggregating points helps mitigate experimental noise and enhances the robustness of downstream analysis \citep{persad2023seacells}. Although DR and clustering share some key similarities, these processes are typically addressed separately, as no specific relationship between them has yet been established.

In \Cref{chapter:DistR}, we propose a novel approach that unifies DR and clustering within a common framework by treating both the data and the embeddings as probability distributions. Through this lens, we demonstrate that dimensionality reduction (DR) approaches can be interpreted as optimal transport problems. Conceptualizing the problem in terms of distributions allows us to control the granularity of the resulting low-dimensional representation, enabling the retrieval of points that act as prototypes for multiple input data points. This formulation allows for the joint computation of a reduced representation and a clustering of the data. We show that, in practice, this approach achieves strong performance in simultaneous clustering and dimensionality reduction, enabling a more effective exploration of the trade-offs between these two aspects.

\paragraph{Chapter 5.}  
In dimensionality reduction (DR) formulations, it is common to represent the structure of the data using a Markov chain matrix, which defines the transition probabilities between points. These row-normalized, or \emph{row-stochastic}, matrices are often symmetrized for DR purposes, typically by averaging them with their transposes, a process known as Euclidean symmetrization. However, in most neighbor embedding methods, the affinity matrix is often preprocessed using pointwise control of Shannon entropy to account for heteroscedastic noise.

In \Cref{chapter:SNEkhorn}, we argue that Euclidean symmetrization may not be suitable in such cases. We demonstrate that symmetrizing under the appropriate geometric framework can resolve this inconsistency, allowing for the simultaneous handling of both Shannon entropy control and symmetrization. We then extend this formulation to a broader class of entropy functions in the context of non-symmetric optimal transport, showing that pointwise constraints can enhance the robustness of the resulting projections in domain adaptation scenarios.
