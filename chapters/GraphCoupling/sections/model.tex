\section{PCA as Graph Coupling}\label{sec:pca_graph_coupling}


We begin by considering the most canonical Markov Random Field (MRF), the multivariate Gaussian model. To emphasize dependencies among the samples, we employ a covariance structure in the form of a Kronecker product between the row and column covariances. For simplicity, we assume an identity matrix for the column covariance, though any covariance structure could be used. For the row precision matrix, we adopt the conjugate prior to this MRF likelihood, which is the well-known Wishart distribution. As we will demonstrate, the natural adaptation of our framework to this model results in the widely recognized PCA embedding solution. In what follows, for a continuous variable $\bm{\Theta}_{Z}$, $\mathbb{P}(\bm{\Theta}_{Z} = \cdot)$ denotes its density.
\todo{give formula of Wishart}

\begin{restatable}{theorem}{PCAgraphcoupling}
\label{PCA_graph_coupling}
Let $\nu \geq n$, $\alpha \geq 0$, $\bm{\Theta}_{X} \sim \mathcal{W}(\nu, \bm{I}_N)$ and $\bm{\Theta}_{Z} \sim \mathcal{W}(\nu + \alpha, \bm{I}_N)$. Assume that $\bm{\Theta}_{X}$ and $\bm{\Theta}_{Z}$ structure the rows of respectively $\Xb$ and $\Zb$ such that: 
\begin{align}
    \mathrm{vec}(\Xb) | \bm{\Theta}_{X} &\sim \mathcal{N}(\bm{0}, \bm{\Theta}_{X}^{-1} \otimes \bm{I}_p), \label{eq:X_given_theta} \\
    \mathrm{vec}(\Zb) | \bm{\Theta}_{Z} &\sim \mathcal{N}(\bm{0}, \bm{\Theta}_{Z}^{-1} \otimes \bm{I}_q) \label{eq:Z_given_theta} \:.
\end{align}
Then the solution to the precision coupling problem:
\begin{align*}
    \min_{\Zb \in \mathbb{R}^{n \times q}} -\mathbb{E}_{\bm{\Theta}_{X} | \Xb}\left[\log \mathbb{P}(\bm{\Theta}_{Z}=\bm{\Theta}_{X}|\Zb)\right]
\end{align*}
is a PCA embedding of $\Xb$ with $q$ components (\Cref{def:PCA_embedding}). When $\alpha = p - q$, the solution is an exact PCA embedding (\Cref{def:PCA_embedding}).
\end{restatable}

Note that unlike usual constructions of PCA or probabilistic PCA \citep{tipping1999probabilistic}, in the above the linear relation between $\Xb$ and $\Zb$ is recovered by solving the graph coupling problem and not explicitly stated beforehand. 

\section {Shift-Invariant Pairwise MRF to Model Sample Dependencies} \label{sec:graph_structure}

We now consider other types of MRFs, characterized by discrete structuring graphs and more general potential functions. We follow the same methodology as in \Cref{sec:pca_graph_coupling}. Specifically, we first define an MRF conditioned on the data given a graph, and then construct a prior for the graph that is conjugate with this conditional distribution.

Let us start by defining the distribution of the observations given a graph. The latter takes the form of a pairwise MRF model which as we show is improper (\textit{i.e.}\ not integrable on $\mathbb{R}^{N \times p}$) when shift-invariant kernels are used. We consider a fixed directed graph $\mW \in \mathcal{S}_{W}$ where:
\begin{align}
    \mathcal{S}_{W} = \left\{\mW \in \mathbb{N}^{N \times N} \mid \forall (i,j) \in \integ{N}^2, W_{ii}=0, W_{ij} \leq n \right\} \:.
\end{align}
Throughout, $(E, \mathcal{B}(E), \lambda_E)$ denotes a measure space where $\mathcal{B}(E)$ is the Borel $\sigma$-algebra on $E$ and $\lambda_E$ is the Lebesgue measure on $E$.

\subsection{Graph Laplacian Null Space}\label{sec:laplacian_prop}
A central element in our construction is the graph Laplacian linear map, defined as follows, where $\mathcal{S}^N_+(\mathbb{R})$ is the set of positive semidefinite matrices.
\begin{definition}\label{graph_laplacian}
The graph Laplacian operator is the map $L \colon \mathbb{R}_+^{N \times N} \cap \mathcal{S}^N(\R) \to \mathcal{S}^N_+(\mathbb{R})$ such that
$$\text{for } (i,j) \in \integ{N}^2, \quad L(\mW)_{ij} = \left\{
\begin{array}{ll}
    - W_{ij} & \text{if } i \neq j \\
    \sum_{k \in \integ{N}} W_{ik} & \text{otherwise} \:.
\end{array} 
\right. $$
\end{definition}
With an abuse of notation, let $\Lb = L(\overline{\mW})$ where $\overline{\mW} = \mW + \mW^\top$. Let $(C_1,...,C_{R})$ be a partition of $\integ{N}$ (\textit{i.e.}\ the set $\{1,2,...,N\}$) corresponding to the connected components (CCs) of $\overline{\mW}$. As well known in spectral graph theory \citep{Chung97}, the null space of $\Lb$ is spanned by the orthonormal vectors $\{\Ub_{r}\}_{r \in [R]}$ such that for $r \in [R]$,
$\Ub_{r} = \left(n_r^{-1/2} \ind_{i \in C_r}\right)_{i \in \integ{N}}$ with $n_r = \operatorname{Card}(C_r)$. By the spectral theorem, $\Ub_{[R]}$ can be completed so that $\Lb = \bm{U \Lambda U^\top}$ where $\Ub = (\Ub_1, ..., \Ub_N)$ is orthogonal and $\bm{\Lambda} = \operatorname{diag}((\lambda_i)_{i \in \integ{N}})$ with $0 = \lambda_1 = ... = \lambda_R < \lambda_{R+1} \leq ... \leq \lambda_Nn$. 

In what follows, the data is split into two parts: $\Xb_{M}$, the orthogonal projection of $\Xb$ on $\mathcal{S}_{M} = (\ker \Lb) \otimes \mathbb{R}^p$, and $\Xb_{C}$, the projection on $\mathcal{S}_{C} = (\ker \Lb)^{\perp} \otimes \mathbb{R}^p$. For $i \in \integ{N}$, $\Xb_{M,i} = \sum_{r \in [R]} n_r^{-1} \ind_{i \in C_r}\sum_{\ell \in C_r} \Xb_{\ell} $ hence $\Xb_{M}$ stands for the empirical means of $\Xb$ on CCs, thus modelling the CC positions, while $\Xb_{C} = \Xb - \Xb_{M}$ is CC-wise centered, thus modeling the relative positions of the nodes within CCs. We now introduce the probability distribution of these variables.

\subsection{Pairwise MRF and Shift-Invariances}\label{sec:within_CC}

The strength of the connection between two nodes is given by a symmetric\footnote{\ie $k(-\vx) = k(\vx)$ for any $\vx$ in $\R^p$.} function $k: \mathbb{R}^p \to \mathbb{R}_+$. We consider the following pairwise MRF unnormalized density function:
\begin{align}\label{eq:unnormalized_MRF}
  f_{k} \colon (\Xb,\mW) &\mapsto \prod_{(i,j) \in \integ{N}^2} k(\mathbf{x}_{i} - \mathbf{x}_{j})^{W_{ij}} \: .
\end{align}
As we will see shortly, the above is at the heart of DR methods based on pairwise similarities. Note that as $k$ measures the similarity between couples of samples, $f_k$ will take high values if the rows of $\Xb$ vary smoothly on the graph $\mW$. Thus we can expect $\mathbf{x}_i$ and $\mathbf{x}_j$ to be close if there is an edge between node $i$ and node $j$ in $\mW$. A key remark is that $f_{k}$ is kept invariant by translating $\Xb_{M}$. Namely for all $\Xb \in \mathbb{R}^{n \times p}$, $f_{k}(\Xb, \mW) = f_{k}(\Xb_{C}, \mW)$. This invariance results in $f_{k}(\cdot, \mW)$ being non integrable on $\mathbb{R}^{n \times p}$, as we see with the following example. 

\paragraph{Gaussian kernel.} For a positive definite matrix $\mathbf{\Sigma} \in \mathcal{S}^N_{++}(\mathbb{R})$, consider the Gaussian kernel $k : \Xb \mapsto e^{- \frac{1}{2}\| \Xb \|_{\mathbf{\Sigma}}^2}$ where $\mathbf{\Sigma}$ stands for the covariance among columns. One has:
\begin{align}\label{eq:gaussian_kernel}
    \log f_{k}(\Xb, \mW) &= -\sum_{(i,j) \in \integ{N}^2} W_{ij} \| \mathbf{x}_{i}-\mathbf{x}_{j} \|^2_{\mathbf{\Sigma}}
    = - \operatorname{tr} \left(\mathbf{\Sigma}^{-1} \Xb^{T} \Lb \Xb\right)
\end{align}
by property of the graph Laplacian (\cref{graph_laplacian}). In this case, it is clear that due to the rank deficiency of $\Lb$, $f_{k}(\cdot, \mW)$ is only $\lambda_{\mathcal{S}_{C}}$-integrable. In general DR settings one does not want to rely on Gaussian kernels only. A striking example is the use of the Student kernel in t-SNE \citep{maaten2008tSNE}. Heavy-tailed kernels appear useful when the dimension of the embeddings is smaller than the intrinsic dimension of the data \citep{kobak2019heavy}. Our contribution provides flexibility by extending the previous result to a large class of kernels, as stated in the following theorem.

\begin{restatable}{theorem}{integrabilitypairwiseMRF}
\label{prop:integrability_pairwise_MRF}
If $k$ is $\lambda_{\mathbb{R}^p}$-integrable and bounded above $\lambda_{\mathbb{R}^p}$-almost everywhere then $f_{k}(\cdot, \mW)$ is $\lambda_{\mathcal{S}_{C}}$-integrable.
\end{restatable}

We refer to \cref{proof:lambda_perp_integrability} for the proof.
We can now define a distribution on $(\mathcal{S}_{C}, \mathcal{B}(\mathcal{S}_{C}))$, where $\mathcal{C}_{k}(\mW) = \int f_{k}(\cdot, \mW) d\lambda_{\mathcal{S}_{C}}$:
\begin{align}\label{eq:proba_perp}
\mathbb{P}_{k}(d\Xb_{C} | \mW) = \mathcal{C}_{k}(\mW)^{-1} f_{k}(\Xb_{C}, \mW) \lambda_{\mathcal{S}_{C}}(d\Xb_{C}) \: .
\end{align}

\begin{remark}
Kernels may have node-specific bandwidths $\bm{\tau}$, set during a pre-processing step, giving $f_{k}(\Xb,\mW) = \prod_{(i,j)} k((\mathbf{x}_{i} - \mathbf{x}_{j})/\tau_{i})^{W_{ij}}$. Note that such bandwidth does not affect the degeneracy of the distribution and \cref{prop:integrability_pairwise_MRF} still holds.
\end{remark}


\paragraph{Between-Rows Dependency Structure.} By symmetry of $k$, reindexing gives: $f_{k}(\Xb, \mW) = \prod_{j \in \integ{N}} \prod_{i \in [j]} k(\mathbf{x}_{i} - \mathbf{x}_{j})^{\overline{W}_{ij}}$. Hence distribution \eqref{eq:proba_perp} boils down to a pairwise MRF model \citep{clifford1990markov} with respect to the undirected graph $\overline{\mW}$, $\mathcal{C}_{k}$ playing the role of the partition function. Note that since $f_k$ (Equation \ref{eq:unnormalized_MRF}) trivially factorize according to the cliques of $\overline{\mW}$, the Hammersley-Clifford theorem ensures that the rows of $\Xb_{C}$ satisfy the local and global Markov properties with respect to $\overline{\mW}$. 

\subsection{Uninformative Model for CC-wise Means}

We showed that the MRF (\ref{eq:unnormalized_MRF}) is only integrable on $\mathcal{S}_{C}$, the definition of which depends on the connectivity structure of $\mW$. As we now demonstrate, the latter MRF can be seen as a limit of proper distributions on $\mathbb{R}^{n \times p}$, see \textit{e.g.}\ \citep{rue2005gaussian} for a similar construction in the Gaussian case. 
We introduce the Borel function $f^{\varepsilon}(\cdot, \mW) \colon \mathbb{R}^{n \times p} \to \mathbb{R}_+$ for $\varepsilon > 0$ such that for all $\Xb \in \mathbb{R}^{n \times p}$, $f^{\varepsilon}(\Xb, \mW) = f^{\varepsilon}(\Xb_{M}, \mW)$. To allow $f^{\varepsilon}$ to become arbitrarily non-informative, we assume that for all $\mW \in \mathcal{S}_{W}$, $f^\varepsilon(\cdot, \mW)$ is $\lambda_{\mathcal{S}_{M}}$-integrable for all $\varepsilon \in \mathbb{R}^*_+$ and $f^{\varepsilon}(\cdot, \mW) \xrightarrow[\varepsilon \to 0]{} 1$ almost everywhere.
We now define the conditional distribution on $(\mathcal{S}_{M}, \mathcal{B}(\mathcal{S}_{M}))$ as follows:
\begin{align}\label{eq:proba_parallel}
     \mathbb{P}^{\varepsilon}(d\Xb_{M}| \mW) = \mathcal{C}^{\varepsilon}(\mW)^{-1} f^{\varepsilon}(\Xb_{M}, \mW) \lambda_{\mathcal{S}_{M}}(d\Xb_{M})
\end{align}
where $\mathcal{C}^{\varepsilon}(\mW) = \int f^{\varepsilon}(\cdot, \mW) d\lambda_{\mathcal{S}_{M}}$.
With this at hand, the joint conditional is defined as the product measure of (\ref{eq:proba_perp}) and (\ref{eq:proba_parallel}) over the row axis, the integrability of which is ensured by the Fubini-Tonelli theorem. In the following we will use the compact notation $\mathcal{C}^{\varepsilon}_k(\mW) = \mathcal{C}_k(\mW)\mathcal{C}^{\varepsilon}(\mW)$ for the joint normalizing constant.

\begin{remark}
At the limit $\varepsilon \to 0$ the above construction amounts to setting an infinite variance on the distribution of the empirical means of $\Xb$ on CCs, thus losing the inter-CC structure. 
\end{remark}

As an illustration, one can structure the CCs' relative positions according to a Gaussian model with positive definite precision $\varepsilon \bm{\Theta} \in \mathcal{S}_{++}^R(\mathbb{R})$, as it amounts to choosing $f^{\varepsilon} : \Xb \to \exp \left(-\frac{\varepsilon}{2} \operatorname{tr}\left(\mathbf{\Sigma}^{-1}\Xb^\top\Ub_{[:R]}  \bm{\Theta}\Ub^\top_{[:R]} \Xb\right)\right)$ such that: $\mathrm{vec}(\Xb_{M}) | \bm{\Theta} \sim \mathcal{N}\left(\bm{0}, \left(\varepsilon \Ub_{[:R]}  \bm{\Theta}\Ub^\top_{[:R]}\right)^{-1} \otimes \mathbf{\Sigma}\right)$ where $\otimes$ denotes the Kronecker product.