\section{Conclusion and Perspectives}\label{Perspectives}

In this work, we shed new light on the most popular DR methods by showing that they can be unified within a common probabilistic model in the form of latent Markov Random Fields Graphs coupled by a cross-entropy. The definition of such a model constitutes a major step towards the understanding of common dimension reduction methods, in particular their structure preservation properties as discussed in this article. 

Our work offers many perspectives, among which the possibility to enrich the probabilistic model with more suited graph priors. Currently considered priors are simply the ones that are conjugate to the MRFs thus they are mostly designed to yield a tractable coupling objective. However they may not be optimal and could be modified to capture targeted features, \textit{e.g.}\ communities, in the input data, and give adapted representations in the latent space. The graph coupling approach could also be extended to more general latent structures governing the joint distribution of observations.
Finally, the probabilistic model could be leveraged to tackle hyper-parameter calibration, especially kernel bandwidths that have a great influence on the quality of the representations and are currently tuned using heuristics with unclear motivations.