\subsection{Optimal Transport Across Spaces : Gromov-Wasserstein Formulation}

\begin{figure}[!t]
	\begin{center}
	\centerline{\includegraphics[width=0.4\columnwidth]{figures/spiral.pdf}}
	\caption{Illustration of the matching obtained from solving \Cref{eq:gw_pb} between two distributions organized as clusters on a spiral: $\nu \in \gP(\mathbb{R}^3)$ and $\mu \in \gP(\mathbb{R}^2)$. The connectivities are based on the Euclidean distances between the locations of the respective distributions. The optimal transport matrix $\mT$ is represented by dashed lines. Taken from \citep{vayer2020contribution}.}
	\label{fig:GW_illustration}
	\end{center}
	\vskip -0.3in
\end{figure}

Classical OT methods require defining a meaningful transportation cost between the supports of the two distributions. 
This is however impossible in the context of dimensionality reduction where the two spaces $\R^p$ and $\R^d$ have different dimensions, and are thus not part of a common ground metric space. We review in this section the Gromov-Wasserstein formulation of OT aiming at comparing distributions \emph{across incomparable spaces}.

In this thesis, we will leverage the GW discrepancy to extend classical DR approaches, framing them as the GW projection of a distribution onto a space of lower dimensionality. 

\paragraph{Gromov-Wasserstein (GW).} The GW framework \citep{memoli2011gromov,sturm2012space} comprises a collection of OT methods designed to compare distributions by examining the pairwise relations \emph{within each domain}. For two matrices $\mP \in \R^{N \times N}$, $\mQ \in \R^{n \times n}$, and weights $\va \in \Sigma_N, \vb \in \Sigma_n$, the GW discrepancy is defined as
\begin{equation}
\label{eq:gw_pb} 
\tag{GW}
\begin{split}
	&\GW_L (\mP, \mQ, \va, \vb) \coloneqq \min_{\mT \in \gU(\va, \vb)} E_L(\mP, \mQ, \mT)\,, \\
	&\text{where} \quad E_L(\mP, \mQ, \mT) \coloneqq \sum_{ijkl}  L(P_{ij}, Q_{kl}) T_{ik} T_{jl} \:.
\end{split}
\end{equation}
% \end{align}
% \begin{align*}
%\sum_{\substack{(i,j) \in \integ{N}^2 \\ (k,l) \in \integ{n}^2}}
%where  $L:\R \times \R \rightarrow \R_+$ is a loss 
% \footnote[1]{our definition of $L_{\mathrm{KL}}$ differs from \citep{peyre2016gromov} where the generalized Kullback-Leibler divergence is considered.}
% and $\gU(\va, \vb) = \left\{ \mT \in \R_+^{N \times n} : \mT \bm{1}_n = \va, \mT^\top \bm{1}_N = \vb \right\}$ is the set of couplings between $\va$ and $\vb$.
In this formulation, both pairs $(\mP, \va)$ and $(\mQ, \vb)$ can be interpreted as graphs with corresponding connectivity matrices $\mP, \overline \mP$, and where nodes are weighted by histograms $\va, \overline \va$. \Cref{eq:gw_pb} is thus a \emph{quadratic problem} (in $\mT$) which consists in finding a soft-assignment matrix $\mT$ that aligns the nodes of the two graphs in a way that preserves their pairwise connectivities. 

\paragraph{Gromov-Wasserstein metric property.}
From a distributional perspective, GW can also be viewed as a distance between distributions that do not belong to the same metric space. For two discrete probability distributions $\mu_X = \sum_{i=1}^N [\va_X]_i \delta_{\vx_i} \in \gP_N(\R^{p}), \mu_Z =\sum_{i=1}^n [\va_Z]_i \delta_{\vz_i} \in \gP_n(\R^d)$ and pairwise similarity matrices $\simiX$ and $\simiZ$ associated with the supports $\mX = (\vx_1, \cdots, \vx_N)^\top$ and $\mZ = (\vz_1, \cdots, \vz_n)^\top$, the quantity $\GW_L (\simiX, \simiZ, \va_X, \va_Z)$ is a measure of dissimilarity or discrepancy between $\mu_X, \mu_Z$. Specifically, when selecting the quadratic loss $L=L_2$, and $\simiX, \simiZ$ are pairwise distance matrices, GW defines a proper distance between $\mu_X$ and $\mu_Z$ (or more specifically their associated metric measured spaces) with respect to measure preserving isometries.

\begin{remark}
    Under weaker assumptions on $\simiX$ and $\simiZ$, \Cref{eq:gw_pb} defines a pseudo-metric with respect to a different notion of isomorphism. Two graphs $(\mP_1, \va_1)$ and $(\mP_2, \va_2)$ are said to be weakly isomorphic \citep{chowdhury2019gromov} if there exists a canonical representation $(\mP_c, \va_c)$ such that $\card(\supp(\va_c)) = p \leq \min(n, m)$ and matrices $\mM_1 \in \left\{0,1 \right\}^{n \times p}$ (resp. $\mM_2 \in \left\{0,1 \right\}^{m \times p}$) satisfying, for $k \in \left\{1, 2\right\}$:
    \begin{equation}
        \mP_k = \mM_k^\top \mP_c \mM_k \quad \text{and} \quad \va_k = \mM_k^\top \va_c \:.
    \end{equation}
\end{remark}


\paragraph{Solving Gromov-Wasserstein.} The objective in \Cref{eq:gw_pb} can be expressed as 
\begin{align}
    \min_{\mT \in \gU(\va, \vb)} \langle \mL(\mP, \mQ) \otimes \mT, \mT \rangle \:,
\end{align}
where $\mL(\mP, \mQ)$ is the tensor $L(P_{ij}, Q_{kl})_{ijkl}$ and $\otimes$ denotes the tensor-matrix multiplication such that $\mL \otimes \mT = (\sum_{kl} L_{ijkl} T_{kl})_{ij}$. The above optimization problem is a non convex quadratic program, which thus belongs to the class of NP-hard problems. 

A naive evaluation of the cost is computationally expensive, requiring $\mathcal{O}(N^2n^2)$ operations. However, as shown in \citep{peyre2016gromov}, when the loss $L$ admits the following decomposition $L(a, b) = f_{1}(a) + f_{2}(b) - h_{1}(a) h_{2}(b)$, which is the case for typical losses such as $\KL$ or square loss, one has
\[
\mL(\mP, \mQ) \otimes \mT = c_{\mP, \mQ} - h_1(\mP) \mT h_2(\mQ)^\top,
\]
where $c_{\mP, \mQ} = f_1(\mP) \va \mathbf{1}_n^\top + \mathbf{1}_N \vb^\top f_2(\mQ)^\top$. This decomposition reduces the computational complexity to $\mathcal{O}(N^2n + n^2N)$ operations.

As we have just demonstrated how to minimize linear objectives over the convex polytope $\gU(\va, \vb)$, it is natural to consider the Frank-Wolfe algorithm \citep{frank1956algorithm} to solve the problem of \Cref{eq:gw_pb}. The algorithm constructs a linear approximation of the objective function and then moves towards the minimizer of this linear function over the feasible domain. This procedure is guaranteed to converge to a local optimum under mild conditions \citep{lacoste2016convergence}. Minimizing the linear approximation corresponds to a classical linear optimal transport (OT) problem. Thus, the algorithm proceeds by iterating the following steps:
\begin{enumerate}
    \item[i)] Compute the gradient of the GW cost evaluated at the current estimate $\mT$:
    \[
        \nabla_{\mT} E_L(\mP, \mQ, \mT) = \mL(\mP, \mQ) \otimes \mT + \mL(\mP^\top, \mQ^\top) \otimes \mT
    \]
    
    \item[ii)] Find a descent direction by minimizing the computed linearization of the problem:
    \[
    \mX^\star \in \argmin_{\mX \in \mathcal{U}(\va, \vb)} \langle \nabla_{\mT} E_L(\mP, \mQ, \mT), \mX \rangle
    \]
    which is a classical linear OT problem, with $\nabla_{\mT} E_L(\mP, \mQ, \mT)$ as the cost matrix.
    
    \item[iii)] Update the current estimate $\mT$ with a line search between $\mT$ and $\mX^\star$, which comes down to a constrained minimization of a second-degree polynomial function that admits a closed-form solution. We refer to \citep{vayer2020contribution} for details on the line search.
\end{enumerate}

Other popular solvers rely on mirror descent strategies. The most notable approaches utilize the $\KL$ geometry, such as the work of \cite{xu2019gromov}, which addresses the unregularized \Cref{eq:gw_pb} objective, or the work of \cite{peyre2016gromov}, which considers the following entropy-regularized problem:
\begin{align}
    \min_{\mT \in \gU(\va, \vb)} E_L(\mP, \mQ, \mT) - \varepsilon^\star \operatorname{H}(\mT) \:.
\end{align}
Interestingly, addressing the latter problem with mirror descent leads to the following update rule:
\begin{align}
    \mT^{(t+1)} \leftarrow \text{Proj}^{\KL}_{\gU(\va, \vb)} \left( \mT^{(t)} \odot e^{-\gamma^{(t)} \left( \nabla_{\mT} E_L(\mP, \mQ, \mT^{(t)}) - \varepsilon^\star \nabla_{\mT} H(\mT^{(t)}) \right)} \right),
\end{align}
where $\gamma^{(t)}$ is the step size at iteration $t$. The projection can then be handled using the Sinkhorn algorithm \citep{cuturi2013sinkhorn}, thus benefiting from the associated computational speedups, as discussed in \Cref{mem:sinkhorn}.

\paragraph{Semi-relaxed formulation.} 
Among the many variants of the GW problem (e.g., unbalanced \citep{sejourne2021unbalanced} or partial \citep{chapel2020partial}), one particularly relevant for this thesis is the semi-relaxed version proposed by \cite{vincent2021semi}. This formulation relaxes the transport polytope $\gU(\va, \vb)$ to $\gU(\va)$, which boils down to optimizing the second marginal of the OT coupling. It is especially well-suited for cases where the second measure is unknown, such as in the context of a projection. Both Frank-Wolfe and mirror descent algorithms can be easily adapted to this setting by projecting onto $\gU(\va)$ instead of $\gU(\va, \vb)$. This adjustment significantly accelerates the algorithm, as minimizing a linear cost over $\gU(\va)$ reduces to a simple argmax, or a soft-argmax when an entropic penalty is used, eliminating the need for an iterative procedure as required for $\gU(\va, \vb)$.


\paragraph{Gromov-Wasserstein for machine learning.}
Due to its versatile properties, notably in comparing distributions over different domains,  the GW problem has found many applications in machine learning, \textit{e.g.,} for 3D meshes alignment \citep{solomon2016entropic,ezuz2017gwcnn}, NLP \citep{alvarez2018gromov}, (co-)clustering  \citep{peyre2016gromov, redko2020co}, single-cell analysis \citep{demetci2020gromov}, neuroimaging \citep{thual2022aligning}, graph representation learning \citep{xu2020gromov, vincent2021online, liu2022robust, vincent2022template, pmlr-v202-zeng23c} and partitioning \citep{xu2019scalable, chowdhury2021generalized}.
